% --------------------------------------------------------------
% This is all preamble stuff that you don't have to worry about.
% Head down to where it says "Start here"
% --------------------------------------------------------------
 
\documentclass[12pt]{article}
 
\usepackage[margin=1in]{geometry} 
\usepackage{amsmath,amsthm,amssymb}
\usepackage[margin=1in]{geometry} 
\usepackage{amsmath,amsthm,amssymb}
\usepackage[T1]{fontenc} %escribe lo del teclado
\usepackage[utf8]{inputenc} %Reconoce algunos símbolos
\usepackage{lmodern} %optimiza algunas fuentes
\usepackage{graphicx}
\graphicspath{ {images/} }
\usepackage{hyperref} % Uso de links
\usepackage{listings}
\usepackage{color}

\definecolor{dkgreen}{rgb}{0,0.6,0}
\definecolor{gray}{rgb}{0.5,0.5,0.5}
\definecolor{mauve}{rgb}{0.58,0,0.82}

\lstset{
  language=Java,
  aboveskip=3mm,
  belowskip=3mm,
  showstringspaces=false,
  columns=flexible,
  basicstyle={\small\ttfamily},
  numbers=none,
  numberstyle=\tiny\color{gray},
  keywordstyle=\color{blue},
  commentstyle=\color{dkgreen},
  stringstyle=\color{mauve},
  breaklines=true,
  breakatwhitespace=true,
  tabsize=3
}

\usepackage{enumitem}

\begin{document}
 
% --------------------------------------------------------------
%                         Start here
% --------------------------------------------------------------
 
\title{CS20B: HW 4}
\author{Nashir Janmohamed\\
Fall, 2019}

\maketitle
\section{Questions}
\begin{enumerate}
  \item Assum strings is an Iterable list of String objects. Using a while loop, list and iterator operations, create code with functionality equivalent to
  \begin{lstlisting}
for (String hold : strings)
    System.out.println(hold);
}
  \end{lstlisting}
  The above code is equivalent to:
    \begin{lstlisting}
Iterator<String> iter = strings.iterator();
while (iter.hasNext()) {
    System.out.println(iter.next());
}
  \end{lstlisting}
  \item For each of the following lists, describe several useful ways they might be sorted.
  \begin{enumerate}
    \item Books
    \begin{enumerate}
      \item Author's last name
      \item Genre
      \item Title of book
      \item Number of pages
    \end{enumerate}
    \item University course descriptions
    \begin{enumerate}
      \item Major
      \item Course Number
      \item Instructor
      \item Course Title
    \end{enumerate}
    \item Professional basketball information
    \begin{enumerate}
      \item Team
      \item Player
      \item Year
    \end{enumerate}
    \item Summer camper registration information
    \begin{enumerate}
      \item Camper Last Name
      \item Camper Age
    \end{enumerate}
    \item Shopping items
    \begin{enumerate}
      \item Price
      \item Type (i.e. fruit, vegetable, dairy)
      \item Store (i.e. where to buy)
    \end{enumerate}
  \end{enumerate}
\end{enumerate}
% --------------------------------------------------------------
%     You don't have to mess with anything below this line.
% --------------------------------------------------------------
 
\end{document}